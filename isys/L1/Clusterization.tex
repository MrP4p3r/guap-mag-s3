


% Default to the notebook output style

    


% Inherit from the specified cell style.




    
\documentclass[12pt,a4paper]{article}

    
    
  
      \usepackage[T1]{fontenc}
    % Nicer default font (+ math font) than Computer Modern for most use cases
    \usepackage{mathpazo}

    % Basic figure setup, for now with no caption control since it's done
    % automatically by Pandoc (which extracts ![](path) syntax from Markdown).
    \usepackage{graphicx}
    % We will generate all images so they have a width \maxwidth. This means
    % that they will get their normal width if they fit onto the page, but
    % are scaled down if they would overflow the margins.
    \makeatletter
    \def\maxwidth{\ifdim\Gin@nat@width>\linewidth\linewidth
    \else\Gin@nat@width\fi}
    \makeatother
    \let\Oldincludegraphics\includegraphics
    % Set max figure width to be 80% of text width, for now hardcoded.
    \renewcommand{\includegraphics}[1]{\Oldincludegraphics[width=.8\maxwidth]{#1}}
    % Ensure that by default, figures have no caption (until we provide a
    % proper Figure object with a Caption API and a way to capture that
    % in the conversion process - todo).
    \usepackage{caption}
    \DeclareCaptionLabelFormat{nolabel}{}
    \captionsetup{labelformat=nolabel}

    \usepackage{adjustbox} % Used to constrain images to a maximum size 
    \usepackage{xcolor} % Allow colors to be defined
    \usepackage{enumerate} % Needed for markdown enumerations to work
    % \usepackage{geometry} % Used to adjust the document margins
    \usepackage{amsmath} % Equations
    \usepackage{amssymb} % Equations
    \usepackage{textcomp} % defines textquotesingle
    % Hack from http://tex.stackexchange.com/a/47451/13684:
    \AtBeginDocument{%
    \def\PYZsq{\textquotesingle}% Upright quotes in Pygmentized code
    }
    \usepackage{upquote} % Upright quotes for verbatim code
    \usepackage{eurosym} % defines \euro
    \usepackage[mathletters]{ucs} % Extended unicode (utf-8) support
    % \usepackage[utf8x]{inputenc} % Allow utf-8 characters in the tex document
    \usepackage{fancyvrb} % verbatim replacement that allows latex
    \usepackage{grffile} % extends the file name processing of package graphics 
	             % to support a larger range 
    % The hyperref package gives us a pdf with properly built
    % internal navigation ('pdf bookmarks' for the table of contents,
    % internal cross-reference links, web links for URLs, etc.)
    \usepackage{hyperref}
    \usepackage{longtable} % longtable support required by pandoc >1.10
    \usepackage{booktabs}  % table support for pandoc > 1.12.2
    \usepackage[inline]{enumitem} % IRkernel/repr support (it uses the enumerate* environment)
    \usepackage[normalem]{ulem} % ulem is needed to support strikethroughs (\sout)
	                    % normalem makes italics be italics, not underlines
  \usepackage{qsettings}


    
  
    
    % Colors for the hyperref package
    \definecolor{urlcolor}{rgb}{0,.145,.698}
    \definecolor{linkcolor}{rgb}{.71,0.21,0.01}
    \definecolor{citecolor}{rgb}{.12,.54,.11}

    % ANSI colors
    \definecolor{ansi-black}{HTML}{3E424D}
    \definecolor{ansi-black-intense}{HTML}{282C36}
    \definecolor{ansi-red}{HTML}{E75C58}
    \definecolor{ansi-red-intense}{HTML}{B22B31}
    \definecolor{ansi-green}{HTML}{00A250}
    \definecolor{ansi-green-intense}{HTML}{007427}
    \definecolor{ansi-yellow}{HTML}{DDB62B}
    \definecolor{ansi-yellow-intense}{HTML}{B27D12}
    \definecolor{ansi-blue}{HTML}{208FFB}
    \definecolor{ansi-blue-intense}{HTML}{0065CA}
    \definecolor{ansi-magenta}{HTML}{D160C4}
    \definecolor{ansi-magenta-intense}{HTML}{A03196}
    \definecolor{ansi-cyan}{HTML}{60C6C8}
    \definecolor{ansi-cyan-intense}{HTML}{258F8F}
    \definecolor{ansi-white}{HTML}{C5C1B4}
    \definecolor{ansi-white-intense}{HTML}{A1A6B2}

    % commands and environments needed by pandoc snippets
    % extracted from the output of `pandoc -s`
    \providecommand{\tightlist}{%
      \setlength{\itemsep}{0pt}\setlength{\parskip}{0pt}}
    \DefineVerbatimEnvironment{Highlighting}{Verbatim}{commandchars=\\\{\}}
    % Add ',fontsize=\small' for more characters per line
    \newenvironment{Shaded}{}{}
    \newcommand{\KeywordTok}[1]{\textcolor[rgb]{0.00,0.44,0.13}{\textbf{{#1}}}}
    \newcommand{\DataTypeTok}[1]{\textcolor[rgb]{0.56,0.13,0.00}{{#1}}}
    \newcommand{\DecValTok}[1]{\textcolor[rgb]{0.25,0.63,0.44}{{#1}}}
    \newcommand{\BaseNTok}[1]{\textcolor[rgb]{0.25,0.63,0.44}{{#1}}}
    \newcommand{\FloatTok}[1]{\textcolor[rgb]{0.25,0.63,0.44}{{#1}}}
    \newcommand{\CharTok}[1]{\textcolor[rgb]{0.25,0.44,0.63}{{#1}}}
    \newcommand{\StringTok}[1]{\textcolor[rgb]{0.25,0.44,0.63}{{#1}}}
    \newcommand{\CommentTok}[1]{\textcolor[rgb]{0.38,0.63,0.69}{\textit{{#1}}}}
    \newcommand{\OtherTok}[1]{\textcolor[rgb]{0.00,0.44,0.13}{{#1}}}
    \newcommand{\AlertTok}[1]{\textcolor[rgb]{1.00,0.00,0.00}{\textbf{{#1}}}}
    \newcommand{\FunctionTok}[1]{\textcolor[rgb]{0.02,0.16,0.49}{{#1}}}
    \newcommand{\RegionMarkerTok}[1]{{#1}}
    \newcommand{\ErrorTok}[1]{\textcolor[rgb]{1.00,0.00,0.00}{\textbf{{#1}}}}
    \newcommand{\NormalTok}[1]{{#1}}
    
    % Additional commands for more recent versions of Pandoc
    \newcommand{\ConstantTok}[1]{\textcolor[rgb]{0.53,0.00,0.00}{{#1}}}
    \newcommand{\SpecialCharTok}[1]{\textcolor[rgb]{0.25,0.44,0.63}{{#1}}}
    \newcommand{\VerbatimStringTok}[1]{\textcolor[rgb]{0.25,0.44,0.63}{{#1}}}
    \newcommand{\SpecialStringTok}[1]{\textcolor[rgb]{0.73,0.40,0.53}{{#1}}}
    \newcommand{\ImportTok}[1]{{#1}}
    \newcommand{\DocumentationTok}[1]{\textcolor[rgb]{0.73,0.13,0.13}{\textit{{#1}}}}
    \newcommand{\AnnotationTok}[1]{\textcolor[rgb]{0.38,0.63,0.69}{\textbf{\textit{{#1}}}}}
    \newcommand{\CommentVarTok}[1]{\textcolor[rgb]{0.38,0.63,0.69}{\textbf{\textit{{#1}}}}}
    \newcommand{\VariableTok}[1]{\textcolor[rgb]{0.10,0.09,0.49}{{#1}}}
    \newcommand{\ControlFlowTok}[1]{\textcolor[rgb]{0.00,0.44,0.13}{\textbf{{#1}}}}
    \newcommand{\OperatorTok}[1]{\textcolor[rgb]{0.40,0.40,0.40}{{#1}}}
    \newcommand{\BuiltInTok}[1]{{#1}}
    \newcommand{\ExtensionTok}[1]{{#1}}
    \newcommand{\PreprocessorTok}[1]{\textcolor[rgb]{0.74,0.48,0.00}{{#1}}}
    \newcommand{\AttributeTok}[1]{\textcolor[rgb]{0.49,0.56,0.16}{{#1}}}
    \newcommand{\InformationTok}[1]{\textcolor[rgb]{0.38,0.63,0.69}{\textbf{\textit{{#1}}}}}
    \newcommand{\WarningTok}[1]{\textcolor[rgb]{0.38,0.63,0.69}{\textbf{\textit{{#1}}}}}
    
    
    % Define a nice break command that doesn't care if a line doesn't already
    % exist.
    \def\br{\hspace*{\fill} \\* }
    % Math Jax compatability definitions
    \def\gt{>}
    \def\lt{<}
    % Document parameters
    \title{Clusterization}
    
    
    

    % Pygments definitions
    
\makeatletter
\def\PY@reset{\let\PY@it=\relax \let\PY@bf=\relax%
    \let\PY@ul=\relax \let\PY@tc=\relax%
    \let\PY@bc=\relax \let\PY@ff=\relax}
\def\PY@tok#1{\csname PY@tok@#1\endcsname}
\def\PY@toks#1+{\ifx\relax#1\empty\else%
    \PY@tok{#1}\expandafter\PY@toks\fi}
\def\PY@do#1{\PY@bc{\PY@tc{\PY@ul{%
    \PY@it{\PY@bf{\PY@ff{#1}}}}}}}
\def\PY#1#2{\PY@reset\PY@toks#1+\relax+\PY@do{#2}}

\expandafter\def\csname PY@tok@w\endcsname{\def\PY@tc##1{\textcolor[rgb]{0.73,0.73,0.73}{##1}}}
\expandafter\def\csname PY@tok@c\endcsname{\let\PY@it=\textit\def\PY@tc##1{\textcolor[rgb]{0.25,0.50,0.50}{##1}}}
\expandafter\def\csname PY@tok@cp\endcsname{\def\PY@tc##1{\textcolor[rgb]{0.74,0.48,0.00}{##1}}}
\expandafter\def\csname PY@tok@k\endcsname{\let\PY@bf=\textbf\def\PY@tc##1{\textcolor[rgb]{0.00,0.50,0.00}{##1}}}
\expandafter\def\csname PY@tok@kp\endcsname{\def\PY@tc##1{\textcolor[rgb]{0.00,0.50,0.00}{##1}}}
\expandafter\def\csname PY@tok@kt\endcsname{\def\PY@tc##1{\textcolor[rgb]{0.69,0.00,0.25}{##1}}}
\expandafter\def\csname PY@tok@o\endcsname{\def\PY@tc##1{\textcolor[rgb]{0.40,0.40,0.40}{##1}}}
\expandafter\def\csname PY@tok@ow\endcsname{\let\PY@bf=\textbf\def\PY@tc##1{\textcolor[rgb]{0.67,0.13,1.00}{##1}}}
\expandafter\def\csname PY@tok@nb\endcsname{\def\PY@tc##1{\textcolor[rgb]{0.00,0.50,0.00}{##1}}}
\expandafter\def\csname PY@tok@nf\endcsname{\def\PY@tc##1{\textcolor[rgb]{0.00,0.00,1.00}{##1}}}
\expandafter\def\csname PY@tok@nc\endcsname{\let\PY@bf=\textbf\def\PY@tc##1{\textcolor[rgb]{0.00,0.00,1.00}{##1}}}
\expandafter\def\csname PY@tok@nn\endcsname{\let\PY@bf=\textbf\def\PY@tc##1{\textcolor[rgb]{0.00,0.00,1.00}{##1}}}
\expandafter\def\csname PY@tok@ne\endcsname{\let\PY@bf=\textbf\def\PY@tc##1{\textcolor[rgb]{0.82,0.25,0.23}{##1}}}
\expandafter\def\csname PY@tok@nv\endcsname{\def\PY@tc##1{\textcolor[rgb]{0.10,0.09,0.49}{##1}}}
\expandafter\def\csname PY@tok@no\endcsname{\def\PY@tc##1{\textcolor[rgb]{0.53,0.00,0.00}{##1}}}
\expandafter\def\csname PY@tok@nl\endcsname{\def\PY@tc##1{\textcolor[rgb]{0.63,0.63,0.00}{##1}}}
\expandafter\def\csname PY@tok@ni\endcsname{\let\PY@bf=\textbf\def\PY@tc##1{\textcolor[rgb]{0.60,0.60,0.60}{##1}}}
\expandafter\def\csname PY@tok@na\endcsname{\def\PY@tc##1{\textcolor[rgb]{0.49,0.56,0.16}{##1}}}
\expandafter\def\csname PY@tok@nt\endcsname{\let\PY@bf=\textbf\def\PY@tc##1{\textcolor[rgb]{0.00,0.50,0.00}{##1}}}
\expandafter\def\csname PY@tok@nd\endcsname{\def\PY@tc##1{\textcolor[rgb]{0.67,0.13,1.00}{##1}}}
\expandafter\def\csname PY@tok@s\endcsname{\def\PY@tc##1{\textcolor[rgb]{0.73,0.13,0.13}{##1}}}
\expandafter\def\csname PY@tok@sd\endcsname{\let\PY@it=\textit\def\PY@tc##1{\textcolor[rgb]{0.73,0.13,0.13}{##1}}}
\expandafter\def\csname PY@tok@si\endcsname{\let\PY@bf=\textbf\def\PY@tc##1{\textcolor[rgb]{0.73,0.40,0.53}{##1}}}
\expandafter\def\csname PY@tok@se\endcsname{\let\PY@bf=\textbf\def\PY@tc##1{\textcolor[rgb]{0.73,0.40,0.13}{##1}}}
\expandafter\def\csname PY@tok@sr\endcsname{\def\PY@tc##1{\textcolor[rgb]{0.73,0.40,0.53}{##1}}}
\expandafter\def\csname PY@tok@ss\endcsname{\def\PY@tc##1{\textcolor[rgb]{0.10,0.09,0.49}{##1}}}
\expandafter\def\csname PY@tok@sx\endcsname{\def\PY@tc##1{\textcolor[rgb]{0.00,0.50,0.00}{##1}}}
\expandafter\def\csname PY@tok@m\endcsname{\def\PY@tc##1{\textcolor[rgb]{0.40,0.40,0.40}{##1}}}
\expandafter\def\csname PY@tok@gh\endcsname{\let\PY@bf=\textbf\def\PY@tc##1{\textcolor[rgb]{0.00,0.00,0.50}{##1}}}
\expandafter\def\csname PY@tok@gu\endcsname{\let\PY@bf=\textbf\def\PY@tc##1{\textcolor[rgb]{0.50,0.00,0.50}{##1}}}
\expandafter\def\csname PY@tok@gd\endcsname{\def\PY@tc##1{\textcolor[rgb]{0.63,0.00,0.00}{##1}}}
\expandafter\def\csname PY@tok@gi\endcsname{\def\PY@tc##1{\textcolor[rgb]{0.00,0.63,0.00}{##1}}}
\expandafter\def\csname PY@tok@gr\endcsname{\def\PY@tc##1{\textcolor[rgb]{1.00,0.00,0.00}{##1}}}
\expandafter\def\csname PY@tok@ge\endcsname{\let\PY@it=\textit}
\expandafter\def\csname PY@tok@gs\endcsname{\let\PY@bf=\textbf}
\expandafter\def\csname PY@tok@gp\endcsname{\let\PY@bf=\textbf\def\PY@tc##1{\textcolor[rgb]{0.00,0.00,0.50}{##1}}}
\expandafter\def\csname PY@tok@go\endcsname{\def\PY@tc##1{\textcolor[rgb]{0.53,0.53,0.53}{##1}}}
\expandafter\def\csname PY@tok@gt\endcsname{\def\PY@tc##1{\textcolor[rgb]{0.00,0.27,0.87}{##1}}}
\expandafter\def\csname PY@tok@err\endcsname{\def\PY@bc##1{\setlength{\fboxsep}{0pt}\fcolorbox[rgb]{1.00,0.00,0.00}{1,1,1}{\strut ##1}}}
\expandafter\def\csname PY@tok@kc\endcsname{\let\PY@bf=\textbf\def\PY@tc##1{\textcolor[rgb]{0.00,0.50,0.00}{##1}}}
\expandafter\def\csname PY@tok@kd\endcsname{\let\PY@bf=\textbf\def\PY@tc##1{\textcolor[rgb]{0.00,0.50,0.00}{##1}}}
\expandafter\def\csname PY@tok@kn\endcsname{\let\PY@bf=\textbf\def\PY@tc##1{\textcolor[rgb]{0.00,0.50,0.00}{##1}}}
\expandafter\def\csname PY@tok@kr\endcsname{\let\PY@bf=\textbf\def\PY@tc##1{\textcolor[rgb]{0.00,0.50,0.00}{##1}}}
\expandafter\def\csname PY@tok@bp\endcsname{\def\PY@tc##1{\textcolor[rgb]{0.00,0.50,0.00}{##1}}}
\expandafter\def\csname PY@tok@fm\endcsname{\def\PY@tc##1{\textcolor[rgb]{0.00,0.00,1.00}{##1}}}
\expandafter\def\csname PY@tok@vc\endcsname{\def\PY@tc##1{\textcolor[rgb]{0.10,0.09,0.49}{##1}}}
\expandafter\def\csname PY@tok@vg\endcsname{\def\PY@tc##1{\textcolor[rgb]{0.10,0.09,0.49}{##1}}}
\expandafter\def\csname PY@tok@vi\endcsname{\def\PY@tc##1{\textcolor[rgb]{0.10,0.09,0.49}{##1}}}
\expandafter\def\csname PY@tok@vm\endcsname{\def\PY@tc##1{\textcolor[rgb]{0.10,0.09,0.49}{##1}}}
\expandafter\def\csname PY@tok@sa\endcsname{\def\PY@tc##1{\textcolor[rgb]{0.73,0.13,0.13}{##1}}}
\expandafter\def\csname PY@tok@sb\endcsname{\def\PY@tc##1{\textcolor[rgb]{0.73,0.13,0.13}{##1}}}
\expandafter\def\csname PY@tok@sc\endcsname{\def\PY@tc##1{\textcolor[rgb]{0.73,0.13,0.13}{##1}}}
\expandafter\def\csname PY@tok@dl\endcsname{\def\PY@tc##1{\textcolor[rgb]{0.73,0.13,0.13}{##1}}}
\expandafter\def\csname PY@tok@s2\endcsname{\def\PY@tc##1{\textcolor[rgb]{0.73,0.13,0.13}{##1}}}
\expandafter\def\csname PY@tok@sh\endcsname{\def\PY@tc##1{\textcolor[rgb]{0.73,0.13,0.13}{##1}}}
\expandafter\def\csname PY@tok@s1\endcsname{\def\PY@tc##1{\textcolor[rgb]{0.73,0.13,0.13}{##1}}}
\expandafter\def\csname PY@tok@mb\endcsname{\def\PY@tc##1{\textcolor[rgb]{0.40,0.40,0.40}{##1}}}
\expandafter\def\csname PY@tok@mf\endcsname{\def\PY@tc##1{\textcolor[rgb]{0.40,0.40,0.40}{##1}}}
\expandafter\def\csname PY@tok@mh\endcsname{\def\PY@tc##1{\textcolor[rgb]{0.40,0.40,0.40}{##1}}}
\expandafter\def\csname PY@tok@mi\endcsname{\def\PY@tc##1{\textcolor[rgb]{0.40,0.40,0.40}{##1}}}
\expandafter\def\csname PY@tok@il\endcsname{\def\PY@tc##1{\textcolor[rgb]{0.40,0.40,0.40}{##1}}}
\expandafter\def\csname PY@tok@mo\endcsname{\def\PY@tc##1{\textcolor[rgb]{0.40,0.40,0.40}{##1}}}
\expandafter\def\csname PY@tok@ch\endcsname{\let\PY@it=\textit\def\PY@tc##1{\textcolor[rgb]{0.25,0.50,0.50}{##1}}}
\expandafter\def\csname PY@tok@cm\endcsname{\let\PY@it=\textit\def\PY@tc##1{\textcolor[rgb]{0.25,0.50,0.50}{##1}}}
\expandafter\def\csname PY@tok@cpf\endcsname{\let\PY@it=\textit\def\PY@tc##1{\textcolor[rgb]{0.25,0.50,0.50}{##1}}}
\expandafter\def\csname PY@tok@c1\endcsname{\let\PY@it=\textit\def\PY@tc##1{\textcolor[rgb]{0.25,0.50,0.50}{##1}}}
\expandafter\def\csname PY@tok@cs\endcsname{\let\PY@it=\textit\def\PY@tc##1{\textcolor[rgb]{0.25,0.50,0.50}{##1}}}

\def\PYZbs{\char`\\}
\def\PYZus{\char`\_}
\def\PYZob{\char`\{}
\def\PYZcb{\char`\}}
\def\PYZca{\char`\^}
\def\PYZam{\char`\&}
\def\PYZlt{\char`\<}
\def\PYZgt{\char`\>}
\def\PYZsh{\char`\#}
\def\PYZpc{\char`\%}
\def\PYZdl{\char`\$}
\def\PYZhy{\char`\-}
\def\PYZsq{\char`\'}
\def\PYZdq{\char`\"}
\def\PYZti{\char`\~}
% for compatibility with earlier versions
\def\PYZat{@}
\def\PYZlb{[}
\def\PYZrb{]}
\makeatother


    % Exact colors from NB
    \definecolor{incolor}{rgb}{0.0, 0.0, 0.5}
    \definecolor{outcolor}{rgb}{0.545, 0.0, 0.0}



  
  \doctype{REP}

  \kafedra{Кафедра вычислительных систем и сетей}

  \profrank{доц., канд. техн. наук}
  \profname{Соловьев Н. В.}

  \docsubject{Кластерный анализ в распознавании образов}
  \docdiscip{Интеллектуальные системы}

  \studclass{4645М}
  \studname{Гетманенко Г. В.}

  \docfooter{\the\year}



    
    % Prevent overflowing lines due to hard-to-break entities
    \sloppy 
    % Setup hyperref package
    \hypersetup{
      breaklinks=true,  % so long urls are correctly broken across lines
      colorlinks=true,
      urlcolor=urlcolor,
      linkcolor=linkcolor,
      citecolor=citecolor,
      }
    % Slightly bigger margins than the latex defaults
    
    

    \begin{document}
    
    
    
  \makeqtitle
  \setcounter{page}{2}

    
    

    
    \section{Задание}\label{ux437ux430ux434ux430ux43dux438ux435}

Вариант 1: Разработать программу, выполняющую кластеризацию пороговым
методом. Для вычисления расстояний между образами необходимо
использовать формулу расстояния Евклида и расстояния Канберра.

    \section{Реализация}\label{ux440ux435ux430ux43bux438ux437ux430ux446ux438ux44f}

Зададим множество точек (образов) для дальнейшей кластеризации.

    \begin{Verbatim}[commandchars=\\\{\}]
{\color{incolor}In [{\color{incolor}1}]:} \PY{o}{\PYZpc{}}\PY{k}{matplotlib} inline
        
        \PY{k+kn}{import} \PY{n+nn}{numpy} \PY{k}{as} \PY{n+nn}{np}
        \PY{k+kn}{import} \PY{n+nn}{matplotlib}\PY{n+nn}{.}\PY{n+nn}{pyplot} \PY{k}{as} \PY{n+nn}{plt}
        \PY{k+kn}{from} \PY{n+nn}{mpl\PYZus{}toolkits}\PY{n+nn}{.}\PY{n+nn}{mplot3d} \PY{k}{import} \PY{n}{Axes3D}
        
        \PY{n}{data} \PY{o}{=} \PY{n}{np}\PY{o}{.}\PY{n}{array}\PY{p}{(}\PY{p}{[}
            \PY{p}{(}\PY{l+m+mi}{0}\PY{p}{,}\PY{l+m+mi}{1}\PY{p}{,}\PY{l+m+mi}{1}\PY{p}{)}\PY{p}{,} \PY{p}{(}\PY{l+m+mi}{0}\PY{p}{,}\PY{l+m+mi}{1}\PY{p}{,}\PY{l+m+mi}{7}\PY{p}{)}\PY{p}{,} \PY{p}{(}\PY{l+m+mi}{5}\PY{p}{,}\PY{l+m+mi}{7}\PY{p}{,}\PY{l+m+mi}{4}\PY{p}{)}\PY{p}{,} \PY{p}{(}\PY{l+m+mi}{0}\PY{p}{,}\PY{l+m+mi}{5}\PY{p}{,}\PY{l+m+mi}{5}\PY{p}{)}\PY{p}{,} \PY{p}{(}\PY{l+m+mi}{9}\PY{p}{,}\PY{l+m+mi}{4}\PY{p}{,}\PY{l+m+mi}{5}\PY{p}{)}\PY{p}{,} \PY{p}{(}\PY{l+m+mi}{7}\PY{p}{,}\PY{l+m+mi}{1}\PY{p}{,}\PY{l+m+mi}{2}\PY{p}{)}\PY{p}{,} \PY{p}{(}\PY{l+m+mi}{10}\PY{p}{,}\PY{l+m+mi}{0}\PY{p}{,}\PY{l+m+mi}{19}\PY{p}{)}\PY{p}{,}
            \PY{p}{(}\PY{l+m+mi}{0}\PY{p}{,}\PY{l+m+mi}{12}\PY{p}{,}\PY{l+m+mi}{7}\PY{p}{)}\PY{p}{,} \PY{p}{(}\PY{o}{\PYZhy{}}\PY{l+m+mi}{5}\PY{p}{,}\PY{o}{\PYZhy{}}\PY{l+m+mi}{4}\PY{p}{,}\PY{l+m+mi}{5}\PY{p}{)}\PY{p}{,} \PY{p}{(}\PY{l+m+mi}{20}\PY{p}{,}\PY{l+m+mi}{10}\PY{p}{,}\PY{l+m+mi}{15}\PY{p}{)}\PY{p}{,} \PY{p}{(}\PY{l+m+mi}{0}\PY{p}{,}\PY{l+m+mi}{16}\PY{p}{,}\PY{o}{\PYZhy{}}\PY{l+m+mi}{16}\PY{p}{)}\PY{p}{,} \PY{p}{(}\PY{o}{\PYZhy{}}\PY{l+m+mi}{1}\PY{p}{,}\PY{l+m+mi}{9}\PY{p}{,}\PY{o}{\PYZhy{}}\PY{l+m+mi}{30}\PY{p}{)}\PY{p}{,}
            \PY{p}{(}\PY{l+m+mi}{18}\PY{p}{,}\PY{l+m+mi}{0}\PY{p}{,}\PY{l+m+mi}{17}\PY{p}{)}\PY{p}{,} \PY{p}{(}\PY{l+m+mi}{6}\PY{p}{,}\PY{l+m+mi}{18}\PY{p}{,}\PY{l+m+mi}{4}\PY{p}{)}
        \PY{p}{]}\PY{p}{,} \PY{n}{dtype}\PY{o}{=}\PY{l+s+s1}{\PYZsq{}}\PY{l+s+s1}{float32}\PY{l+s+s1}{\PYZsq{}}\PY{p}{)}
        
        \PY{n}{x}\PY{p}{,} \PY{n}{y}\PY{p}{,} \PY{n}{z} \PY{o}{=} \PY{n}{data}\PY{o}{.}\PY{n}{T}
        \PY{n}{fig} \PY{o}{=} \PY{n}{plt}\PY{o}{.}\PY{n}{figure}\PY{p}{(}\PY{n}{figsize}\PY{o}{=}\PY{p}{(}\PY{l+m+mi}{6}\PY{p}{,} \PY{l+m+mi}{6}\PY{p}{)}\PY{p}{)}
        \PY{n}{ax} \PY{o}{=} \PY{n}{fig}\PY{o}{.}\PY{n}{add\PYZus{}subplot}\PY{p}{(}\PY{l+m+mi}{111}\PY{p}{,} \PY{n}{projection}\PY{o}{=}\PY{l+s+s1}{\PYZsq{}}\PY{l+s+s1}{3d}\PY{l+s+s1}{\PYZsq{}}\PY{p}{)}
        \PY{n}{ax}\PY{o}{.}\PY{n}{scatter}\PY{p}{(}\PY{n}{x}\PY{p}{,} \PY{n}{y}\PY{p}{,} \PY{n}{z}\PY{p}{,} \PY{n}{c}\PY{o}{=}\PY{l+s+s1}{\PYZsq{}}\PY{l+s+s1}{b}\PY{l+s+s1}{\PYZsq{}}\PY{p}{,} \PY{n}{marker}\PY{o}{=}\PY{l+s+s1}{\PYZsq{}}\PY{l+s+s1}{o}\PY{l+s+s1}{\PYZsq{}}\PY{p}{,} \PY{n}{s}\PY{o}{=}\PY{l+m+mi}{100}\PY{p}{)}
        \PY{n}{plt}\PY{o}{.}\PY{n}{show}\PY{p}{(}\PY{p}{)}
\end{Verbatim}


    \begin{center}
        \adjustimage{max size={0.6\linewidth}{0.6\paperheight}}{Clusterization_files/Clusterization_2_0.png}
    \end{center}
    { \hspace*{\fill} \\}
    
    Функции для расчета расстояния между точками. Используется расстояние
Евклида:

\[ \sqrt{\sum_{i=1}^{n} (x_i - y_i)^2}. \]

И расстояние Канберра:

\[ \sum_{i=1}^{n} \left|\dfrac{x_i - y_i}{\left| x_i \right| + \left| y_i \right|}\right| \]

При вычислении расстояния Канберра возможно деление на ноль, поэтому все
нули в знаменателе заменяются на \(10^{-5}\).

    \begin{Verbatim}[commandchars=\\\{\}]
{\color{incolor}In [{\color{incolor}2}]:} \PY{k}{def} \PY{n+nf}{euclidian\PYZus{}distances}\PY{p}{(}\PY{n}{x}\PY{p}{,} \PY{n}{y}\PY{p}{,} \PY{n}{p}\PY{o}{=}\PY{l+m+mi}{2}\PY{p}{)}\PY{p}{:}
            \PY{k}{return} \PY{n}{np}\PY{o}{.}\PY{n}{sqrt}\PY{p}{(}\PY{n}{np}\PY{o}{.}\PY{n}{sum}\PY{p}{(}\PY{p}{(}\PY{n}{x} \PY{o}{\PYZhy{}} \PY{n}{y}\PY{p}{)}\PY{o}{*}\PY{o}{*}\PY{n}{p}\PY{p}{,} \PY{n}{axis}\PY{o}{=}\PY{l+m+mi}{1}\PY{p}{)}\PY{p}{)}
        
        \PY{k}{def} \PY{n+nf}{canberra\PYZus{}distances}\PY{p}{(}\PY{n}{x}\PY{p}{,} \PY{n}{y}\PY{p}{)}\PY{p}{:}
            \PY{n}{divisor} \PY{o}{=} \PY{n}{np}\PY{o}{.}\PY{n}{abs}\PY{p}{(}\PY{n}{x}\PY{p}{)}\PY{o}{+}\PY{n}{np}\PY{o}{.}\PY{n}{abs}\PY{p}{(}\PY{n}{y}\PY{p}{)}
            \PY{n}{np}\PY{o}{.}\PY{n}{place}\PY{p}{(}\PY{n}{divisor}\PY{p}{,} \PY{p}{(}\PY{n}{divisor} \PY{o}{==} \PY{l+m+mi}{0}\PY{p}{)}\PY{p}{,} \PY{l+m+mf}{1e\PYZhy{}5}\PY{p}{)}
            \PY{k}{return} \PY{n}{np}\PY{o}{.}\PY{n}{sum}\PY{p}{(}\PY{n}{np}\PY{o}{.}\PY{n}{abs}\PY{p}{(}\PY{p}{(}\PY{n}{x}\PY{o}{\PYZhy{}}\PY{n}{y}\PY{p}{)}\PY{o}{/}\PY{n}{divisor}\PY{p}{)}\PY{p}{,} \PY{n}{axis}\PY{o}{=}\PY{l+m+mi}{1}\PY{p}{)}
\end{Verbatim}


    Реализуем функцию кластеризации в соответствии выбранным методом и пару
вспомогательных функций

    \begin{Verbatim}[commandchars=\\\{\}]
{\color{incolor}In [{\color{incolor}3}]:} \PY{k}{def} \PY{n+nf}{calculate\PYZus{}distances}\PY{p}{(}\PY{n}{func}\PY{p}{,} \PY{n}{image}\PY{p}{,} \PY{n}{clusters}\PY{p}{)}\PY{p}{:}
            \PY{l+s+sd}{\PYZdq{}\PYZdq{}\PYZdq{}Вспомогательная функция для вычисления расстояния между}
        \PY{l+s+sd}{    заданным образом и кластерами\PYZdq{}\PYZdq{}\PYZdq{}}
            \PY{n}{image\PYZus{}t} \PY{o}{=} \PY{n}{np}\PY{o}{.}\PY{n}{tile}\PY{p}{(}\PY{n}{image}\PY{p}{,} \PY{p}{(}\PY{n}{clusters}\PY{o}{.}\PY{n}{shape}\PY{p}{[}\PY{l+m+mi}{0}\PY{p}{]}\PY{p}{,} \PY{l+m+mi}{1}\PY{p}{)}\PY{p}{)}
            \PY{n}{distance} \PY{o}{=} \PY{n}{func}\PY{p}{(}\PY{n}{image\PYZus{}t}\PY{p}{,} \PY{n}{clusters}\PY{p}{)}
            \PY{k}{return} \PY{n}{distance}
        
        \PY{k}{def} \PY{n+nf}{calculate\PYZus{}cluster\PYZus{}center}\PY{p}{(}\PY{n}{cluster}\PY{p}{)}\PY{p}{:}
            \PY{l+s+sd}{\PYZdq{}\PYZdq{}\PYZdq{}Вычисляет арифмитическое среднее\PYZdq{}\PYZdq{}\PYZdq{}}
            \PY{k}{return} \PY{n}{np}\PY{o}{.}\PY{n}{sum}\PY{p}{(}\PY{n}{cluster}\PY{p}{,} \PY{n}{axis}\PY{o}{=}\PY{l+m+mi}{0}\PY{p}{)}\PY{o}{/}\PY{n}{cluster}\PY{o}{.}\PY{n}{shape}\PY{p}{[}\PY{l+m+mi}{0}\PY{p}{]}
        
        \PY{k}{def} \PY{n+nf}{clusterize}\PY{p}{(}\PY{n}{data}\PY{p}{,} \PY{n}{threshold}\PY{p}{,} \PY{n}{func}\PY{p}{)}\PY{p}{:}
            \PY{l+s+sd}{\PYZdq{}\PYZdq{}\PYZdq{}Выполняет кластеризацию данных, используя заданный порог}
        \PY{l+s+sd}{    и функцию вычисления расстояния\PYZdq{}\PYZdq{}\PYZdq{}}
            \PY{n}{clusters\PYZus{}centers} \PY{o}{=} \PY{n}{data}\PY{p}{[}\PY{p}{:}\PY{l+m+mi}{1}\PY{p}{]}\PY{o}{.}\PY{n}{copy}\PY{p}{(}\PY{p}{)}
            \PY{n}{clusters\PYZus{}images} \PY{o}{=} \PY{p}{[}\PY{n}{data}\PY{p}{[}\PY{p}{:}\PY{l+m+mi}{1}\PY{p}{]}\PY{o}{.}\PY{n}{copy}\PY{p}{(}\PY{p}{)}\PY{p}{]}
            \PY{k}{for} \PY{n}{image} \PY{o+ow}{in} \PY{n}{data}\PY{p}{[}\PY{l+m+mi}{1}\PY{p}{:}\PY{p}{]}\PY{p}{:}
                \PY{n}{distances} \PY{o}{=} \PY{n}{calculate\PYZus{}distances}\PY{p}{(}\PY{n}{func}\PY{p}{,} \PY{n}{image}\PY{p}{,} \PY{n}{clusters\PYZus{}centers}\PY{p}{)}
                \PY{k}{if} \PY{n}{np}\PY{o}{.}\PY{n}{all}\PY{p}{(}\PY{n}{distances} \PY{o}{\PYZgt{}} \PY{n}{threshold}\PY{p}{)}\PY{p}{:}
                    \PY{c+c1}{\PYZsh{} новый кластер}
                    \PY{n}{clusters\PYZus{}centers} \PY{o}{=} \PY{n}{np}\PY{o}{.}\PY{n}{append}\PY{p}{(}
                        \PY{n}{clusters\PYZus{}centers}\PY{p}{,} \PY{n}{np}\PY{o}{.}\PY{n}{array}\PY{p}{(}\PY{p}{[}\PY{n}{image}\PY{o}{.}\PY{n}{copy}\PY{p}{(}\PY{p}{)}\PY{p}{]}\PY{p}{)}\PY{p}{,} \PY{n}{axis}\PY{o}{=}\PY{l+m+mi}{0}\PY{p}{)}
                    \PY{n}{clusters\PYZus{}images}\PY{o}{.}\PY{n}{append}\PY{p}{(}\PY{n}{np}\PY{o}{.}\PY{n}{array}\PY{p}{(}\PY{p}{[}\PY{n}{image}\PY{o}{.}\PY{n}{copy}\PY{p}{(}\PY{p}{)}\PY{p}{]}\PY{p}{)}\PY{p}{)}
                \PY{k}{else}\PY{p}{:}
                    \PY{c+c1}{\PYZsh{} добавляю в существующий кластер}
                    \PY{n}{idx} \PY{o}{=} \PY{n}{np}\PY{o}{.}\PY{n}{argmin}\PY{p}{(}\PY{n}{distances}\PY{p}{)}
                    \PY{n}{newcluster} \PY{o}{=} \PY{n}{np}\PY{o}{.}\PY{n}{append}\PY{p}{(}
                        \PY{n}{clusters\PYZus{}images}\PY{p}{[}\PY{n}{idx}\PY{p}{]}\PY{p}{,}
                        \PY{n}{np}\PY{o}{.}\PY{n}{array}\PY{p}{(}\PY{p}{[}\PY{n}{image}\PY{o}{.}\PY{n}{copy}\PY{p}{(}\PY{p}{)}\PY{p}{]}\PY{p}{)}\PY{p}{,} \PY{n}{axis}\PY{o}{=}\PY{l+m+mi}{0}\PY{p}{)}
                    \PY{n}{clusters\PYZus{}images}\PY{p}{[}\PY{n}{idx}\PY{p}{]} \PY{o}{=} \PY{n}{newcluster}
                    \PY{n}{clusters\PYZus{}centers}\PY{p}{[}\PY{n}{idx}\PY{p}{]} \PY{o}{=} \PY{n}{calculate\PYZus{}cluster\PYZus{}center}\PY{p}{(}\PY{n}{newcluster}\PY{p}{)}
            \PY{c+c1}{\PYZsh{} форматирование данных}
            \PY{n}{number\PYZus{}of\PYZus{}clusters} \PY{o}{=} \PY{n+nb}{len}\PY{p}{(}\PY{n}{clusters\PYZus{}centers}\PY{p}{)}
            \PY{n}{result} \PY{o}{=} \PY{n}{np}\PY{o}{.}\PY{n}{zeros}\PY{p}{(}\PY{p}{(}\PY{n}{number\PYZus{}of\PYZus{}clusters}\PY{p}{,} \PY{l+m+mi}{2}\PY{p}{)}\PY{p}{,} \PY{n}{dtype}\PY{o}{=}\PY{l+s+s1}{\PYZsq{}}\PY{l+s+s1}{O}\PY{l+s+s1}{\PYZsq{}}\PY{p}{)}
            \PY{k}{for} \PY{n}{i} \PY{o+ow}{in} \PY{n+nb}{range}\PY{p}{(}\PY{n}{number\PYZus{}of\PYZus{}clusters}\PY{p}{)}\PY{p}{:}
                \PY{n}{result}\PY{p}{[}\PY{n}{i}\PY{p}{,}\PY{l+m+mi}{0}\PY{p}{]} \PY{o}{=} \PY{n}{clusters\PYZus{}centers}\PY{p}{[}\PY{n}{i}\PY{p}{]}
                \PY{n}{result}\PY{p}{[}\PY{n}{i}\PY{p}{,}\PY{l+m+mi}{1}\PY{p}{]} \PY{o}{=} \PY{n}{clusters\PYZus{}images}\PY{p}{[}\PY{n}{i}\PY{p}{]}
            \PY{k}{return} \PY{n}{result}
\end{Verbatim}

\newpage
    Теперь можно поэксперементировать с кластеризацией, выбирая предел и
функцию расстояния.

    \begin{Verbatim}[commandchars=\\\{\}]
{\color{incolor}In [{\color{incolor}4}]:} \PY{k+kn}{from} \PY{n+nn}{pprint} \PY{k}{import} \PY{n}{pprint}
        
        \PY{k}{def} \PY{n+nf}{generate\PYZus{}colors\PYZus{}and\PYZus{}markers}\PY{p}{(}\PY{n}{n}\PY{p}{)}\PY{p}{:}
            \PY{n}{markers} \PY{o}{=} \PY{l+s+s1}{\PYZsq{}}\PY{l+s+s1}{o\PYZca{}*hDsv\PYZlt{}\PYZgt{}x+.}\PY{l+s+s1}{\PYZsq{}}
            \PY{n}{result} \PY{o}{=} \PY{p}{[}\PY{p}{]}
            \PY{n}{to\PYZus{}hex} \PY{o}{=} \PY{k}{lambda} \PY{n}{x}\PY{p}{:} \PY{n}{x}\PY{o}{.}\PY{n}{astype}\PY{p}{(}\PY{l+s+s1}{\PYZsq{}}\PY{l+s+s1}{uint8}\PY{l+s+s1}{\PYZsq{}}\PY{p}{)}\PY{o}{.}\PY{n}{tobytes}\PY{p}{(}\PY{p}{)}\PY{o}{.}\PY{n}{hex}\PY{p}{(}\PY{p}{)}
            \PY{k}{for} \PY{n}{i} \PY{o+ow}{in} \PY{n+nb}{range}\PY{p}{(}\PY{n}{n}\PY{p}{)}\PY{p}{:}
                \PY{n}{base\PYZus{}color} \PY{o}{=} \PY{p}{(}\PY{l+m+mi}{42}\PY{o}{+}\PY{n}{np}\PY{o}{.}\PY{n}{random}\PY{o}{.}\PY{n}{random}\PY{p}{(}\PY{l+m+mi}{3}\PY{p}{)}\PY{o}{*}\PY{l+m+mi}{171}\PY{p}{)}
                \PY{n}{cluster\PYZus{}center\PYZus{}color} \PY{o}{=} \PY{l+s+s1}{\PYZsq{}}\PY{l+s+s1}{\PYZsh{}}\PY{l+s+s1}{\PYZsq{}} \PY{o}{+} \PY{n}{to\PYZus{}hex}\PY{p}{(}\PY{n}{base\PYZus{}color} \PY{o}{*} \PY{l+m+mf}{1.05}\PY{p}{)}
                \PY{n}{point\PYZus{}color} \PY{o}{=} \PY{l+s+s1}{\PYZsq{}}\PY{l+s+s1}{\PYZsh{}}\PY{l+s+s1}{\PYZsq{}} \PY{o}{+} \PY{n}{to\PYZus{}hex}\PY{p}{(}\PY{n}{base\PYZus{}color}\PY{p}{)}
                \PY{n}{result}\PY{o}{.}\PY{n}{append}\PY{p}{(}\PY{p}{(}\PY{n}{cluster\PYZus{}center\PYZus{}color}\PY{p}{,} \PY{n}{point\PYZus{}color}\PY{p}{,} \PY{n}{markers}\PY{p}{[}\PY{n}{i}\PY{p}{]}\PY{p}{)}\PY{p}{)}
            \PY{k}{return} \PY{n}{result}
        
        \PY{k}{def} \PY{n+nf}{try\PYZus{}clusterize}\PY{p}{(}\PY{n}{threshold}\PY{p}{,} \PY{n}{func}\PY{p}{)}\PY{p}{:}
            \PY{n}{clusterized\PYZus{}data} \PY{o}{=} \PY{n}{clusterize}\PY{p}{(}\PY{n}{data}\PY{p}{,} \PY{n}{threshold}\PY{p}{,} \PY{n}{func}\PY{p}{)}
            \PY{c+c1}{\PYZsh{} dict repr}
            \PY{n}{dict\PYZus{}repr} \PY{o}{=} \PY{p}{\PYZob{}}
                \PY{n+nb}{tuple}\PY{p}{(}\PY{n+nb}{map}\PY{p}{(}\PY{p}{(}\PY{k}{lambda} \PY{n}{p}\PY{p}{:} \PY{n+nb}{round}\PY{p}{(}\PY{n}{p}\PY{p}{,} \PY{l+m+mi}{2}\PY{p}{)}\PY{p}{)}\PY{p}{,} \PY{n}{x}\PY{p}{[}\PY{l+m+mi}{0}\PY{p}{]}\PY{o}{.}\PY{n}{tolist}\PY{p}{(}\PY{p}{)}\PY{p}{)}\PY{p}{)}\PY{p}{:}
                    \PY{n+nb}{list}\PY{p}{(}\PY{n+nb}{map}\PY{p}{(}\PY{n+nb}{tuple}\PY{p}{,} \PY{n}{x}\PY{p}{[}\PY{l+m+mi}{1}\PY{p}{]}\PY{o}{.}\PY{n}{tolist}\PY{p}{(}\PY{p}{)}\PY{p}{)}\PY{p}{)}
                \PY{k}{for} \PY{n}{x} \PY{o+ow}{in} \PY{n}{clusterized\PYZus{}data}
            \PY{p}{\PYZcb{}}
            \PY{n}{pprint}\PY{p}{(}\PY{n}{dict\PYZus{}repr}\PY{p}{,} \PY{n}{width}\PY{o}{=}\PY{l+m+mi}{65}\PY{p}{)}
            \PY{c+c1}{\PYZsh{} 3d plot}
            \PY{n}{fig} \PY{o}{=} \PY{n}{plt}\PY{o}{.}\PY{n}{figure}\PY{p}{(}\PY{n}{figsize}\PY{o}{=}\PY{p}{(}\PY{l+m+mi}{6}\PY{p}{,} \PY{l+m+mi}{6}\PY{p}{)}\PY{p}{)}
            \PY{n}{ax} \PY{o}{=} \PY{n}{fig}\PY{o}{.}\PY{n}{add\PYZus{}subplot}\PY{p}{(}\PY{l+m+mi}{111}\PY{p}{,} \PY{n}{projection}\PY{o}{=}\PY{l+s+s1}{\PYZsq{}}\PY{l+s+s1}{3d}\PY{l+s+s1}{\PYZsq{}}\PY{p}{)}
            \PY{n}{cms} \PY{o}{=} \PY{n}{generate\PYZus{}colors\PYZus{}and\PYZus{}markers}\PY{p}{(}\PY{n}{clusterized\PYZus{}data}\PY{o}{.}\PY{n}{shape}\PY{p}{[}\PY{l+m+mi}{0}\PY{p}{]}\PY{p}{)}
            \PY{k}{for} \PY{n}{cluster}\PY{p}{,} \PY{n}{cm} \PY{o+ow}{in} \PY{n+nb}{zip}\PY{p}{(}\PY{n}{clusterized\PYZus{}data}\PY{p}{,} \PY{n}{cms}\PY{p}{)}\PY{p}{:}
                \PY{n}{cx}\PY{p}{,} \PY{n}{cy}\PY{p}{,} \PY{n}{cz} \PY{o}{=} \PY{n}{cluster}\PY{p}{[}\PY{l+m+mi}{0}\PY{p}{]}\PY{o}{.}\PY{n}{T}
                \PY{n}{x}\PY{p}{,} \PY{n}{y}\PY{p}{,} \PY{n}{z} \PY{o}{=} \PY{n}{cluster}\PY{p}{[}\PY{l+m+mi}{1}\PY{p}{]}\PY{o}{.}\PY{n}{T}
                \PY{n}{ax}\PY{o}{.}\PY{n}{scatter}\PY{p}{(}\PY{n}{cx}\PY{p}{,} \PY{n}{cy}\PY{p}{,} \PY{n}{cz}\PY{p}{,} \PY{n}{c}\PY{o}{=}\PY{n}{cm}\PY{p}{[}\PY{l+m+mi}{0}\PY{p}{]}\PY{p}{,} \PY{n}{marker}\PY{o}{=}\PY{n}{cm}\PY{p}{[}\PY{l+m+mi}{2}\PY{p}{]}\PY{p}{,} \PY{n}{s}\PY{o}{=}\PY{l+m+mi}{1000}\PY{p}{,} \PY{n}{alpha}\PY{o}{=}\PY{l+m+mf}{0.3}\PY{p}{)}
                \PY{n}{ax}\PY{o}{.}\PY{n}{scatter}\PY{p}{(}\PY{n}{x}\PY{p}{,} \PY{n}{y}\PY{p}{,} \PY{n}{z}\PY{p}{,} \PY{n}{c}\PY{o}{=}\PY{n}{cm}\PY{p}{[}\PY{l+m+mi}{1}\PY{p}{]}\PY{p}{,} \PY{n}{marker}\PY{o}{=}\PY{n}{cm}\PY{p}{[}\PY{l+m+mi}{2}\PY{p}{]}\PY{p}{,} \PY{n}{s}\PY{o}{=}\PY{l+m+mi}{30}\PY{p}{,} \PY{n}{alpha}\PY{o}{=}\PY{l+m+mi}{1}\PY{p}{)}
            \PY{n}{plt}\PY{o}{.}\PY{n}{show}\PY{p}{(}\PY{p}{)}
\end{Verbatim}

\newpage
    Расстояние Евклида.

    \begin{Verbatim}[commandchars=\\\{\}]
{\color{incolor}In [{\color{incolor}5}]:} \PY{n}{try\PYZus{}clusterize}\PY{p}{(}\PY{l+m+mi}{7}\PY{p}{,} \PY{n}{euclidian\PYZus{}distances}\PY{p}{)}
\end{Verbatim}


    \begin{Verbatim}[commandchars=\\\{\}]
\{(-5.0, -4.0, 5.0): [(-5.0, -4.0, 5.0)],
 (-1.0, 9.0, -30.0): [(-1.0, 9.0, -30.0)],
 (0.0, 2.33, 4.33): [(0.0, 1.0, 1.0),
                     (0.0, 1.0, 7.0),
                     (0.0, 5.0, 5.0)],
 (0.0, 12.0, 7.0): [(0.0, 12.0, 7.0)],
 (0.0, 16.0, -16.0): [(0.0, 16.0, -16.0)],
 (6.0, 18.0, 4.0): [(6.0, 18.0, 4.0)],
 (7.0, 4.0, 3.67): [(5.0, 7.0, 4.0),
                    (9.0, 4.0, 5.0),
                    (7.0, 1.0, 2.0)],
 (10.0, 0.0, 19.0): [(10.0, 0.0, 19.0)],
 (18.0, 0.0, 17.0): [(18.0, 0.0, 17.0)],
 (20.0, 10.0, 15.0): [(20.0, 10.0, 15.0)]\}

    \end{Verbatim}

    \begin{center}
        \adjustimage{max size={0.6\linewidth}{0.6\paperheight}}{Clusterization_files/Clusterization_12_1.png}
    \end{center}
    { \hspace*{\fill} \\}
    \newpage
    \begin{Verbatim}[commandchars=\\\{\}]
{\color{incolor}In [{\color{incolor}6}]:} \PY{n}{try\PYZus{}clusterize}\PY{p}{(}\PY{l+m+mi}{12}\PY{p}{,} \PY{n}{euclidian\PYZus{}distances}\PY{p}{)}
\end{Verbatim}


    \begin{Verbatim}[commandchars=\\\{\}]
\{(-1.0, 9.0, -30.0): [(-1.0, 9.0, -30.0)],
 (0.0, 16.0, -16.0): [(0.0, 16.0, -16.0)],
 (2.0, 3.38, 4.5): [(0.0, 1.0, 1.0),
                    (0.0, 1.0, 7.0),
                    (5.0, 7.0, 4.0),
                    (0.0, 5.0, 5.0),
                    (9.0, 4.0, 5.0),
                    (7.0, 1.0, 2.0),
                    (0.0, 12.0, 7.0),
                    (-5.0, -4.0, 5.0)],
 (6.0, 18.0, 4.0): [(6.0, 18.0, 4.0)],
 (14.0, 0.0, 18.0): [(10.0, 0.0, 19.0), (18.0, 0.0, 17.0)],
 (20.0, 10.0, 15.0): [(20.0, 10.0, 15.0)]\}

    \end{Verbatim}

    \begin{center}
        \adjustimage{max size={0.6\linewidth}{0.6\paperheight}}{Clusterization_files/Clusterization_14_1.png}
    \end{center}
    { \hspace*{\fill} \\}
    \newpage
    \begin{Verbatim}[commandchars=\\\{\}]
{\color{incolor}In [{\color{incolor}7}]:} \PY{n}{try\PYZus{}clusterize}\PY{p}{(}\PY{l+m+mi}{17}\PY{p}{,} \PY{n}{euclidian\PYZus{}distances}\PY{p}{)}
\end{Verbatim}


    \begin{Verbatim}[commandchars=\\\{\}]
\{(-0.5, 12.5, -23.0): [(0.0, 16.0, -16.0), (-1.0, 9.0, -30.0)],
 (3.2, 4.5, 5.9): [(0.0, 1.0, 1.0),
                   (0.0, 1.0, 7.0),
                   (5.0, 7.0, 4.0),
                   (0.0, 5.0, 5.0),
                   (9.0, 4.0, 5.0),
                   (7.0, 1.0, 2.0),
                   (10.0, 0.0, 19.0),
                   (0.0, 12.0, 7.0),
                   (-5.0, -4.0, 5.0),
                   (6.0, 18.0, 4.0)],
 (19.0, 5.0, 16.0): [(20.0, 10.0, 15.0), (18.0, 0.0, 17.0)]\}

    \end{Verbatim}

    \begin{center}
        \adjustimage{max size={0.6\linewidth}{0.6\paperheight}}{Clusterization_files/Clusterization_16_1.png}
    \end{center}
    { \hspace*{\fill} \\}
    \newpage
    Расстояние Канберра.

    \begin{Verbatim}[commandchars=\\\{\}]
{\color{incolor}In [{\color{incolor}8}]:} \PY{n}{try\PYZus{}clusterize}\PY{p}{(}\PY{l+m+mf}{1.2}\PY{p}{,} \PY{n}{canberra\PYZus{}distances}\PY{p}{)}
\end{Verbatim}


    \begin{Verbatim}[commandchars=\\\{\}]
\{(-5.0, -4.0, 5.0): [(-5.0, -4.0, 5.0)],
 (-1.0, 9.0, -30.0): [(-1.0, 9.0, -30.0)],
 (0.0, 4.75, 5.0): [(0.0, 1.0, 1.0),
                    (0.0, 1.0, 7.0),
                    (0.0, 5.0, 5.0),
                    (0.0, 12.0, 7.0)],
 (0.0, 16.0, -16.0): [(0.0, 16.0, -16.0)],
 (6.75, 7.5, 3.75): [(5.0, 7.0, 4.0),
                     (9.0, 4.0, 5.0),
                     (7.0, 1.0, 2.0),
                     (6.0, 18.0, 4.0)],
 (14.0, 0.0, 18.0): [(10.0, 0.0, 19.0), (18.0, 0.0, 17.0)],
 (20.0, 10.0, 15.0): [(20.0, 10.0, 15.0)]\}

    \end{Verbatim}

    \begin{center}
        \adjustimage{max size={0.6\linewidth}{0.6\paperheight}}{Clusterization_files/Clusterization_19_1.png}
    \end{center}
    { \hspace*{\fill} \\}
    \newpage
    \begin{Verbatim}[commandchars=\\\{\}]
{\color{incolor}In [{\color{incolor}9}]:} \PY{n}{try\PYZus{}clusterize}\PY{p}{(}\PY{l+m+mf}{2.1}\PY{p}{,} \PY{n}{canberra\PYZus{}distances}\PY{p}{)}
\end{Verbatim}


    \begin{Verbatim}[commandchars=\\\{\}]
\{(-0.5, 12.5, -23.0): [(0.0, 16.0, -16.0), (-1.0, 9.0, -30.0)],
 (2.44, 5.0, 4.44): [(0.0, 1.0, 1.0),
                     (0.0, 1.0, 7.0),
                     (5.0, 7.0, 4.0),
                     (0.0, 5.0, 5.0),
                     (9.0, 4.0, 5.0),
                     (7.0, 1.0, 2.0),
                     (0.0, 12.0, 7.0),
                     (-5.0, -4.0, 5.0),
                     (6.0, 18.0, 4.0)],
 (16.0, 3.33, 17.0): [(10.0, 0.0, 19.0),
                      (20.0, 10.0, 15.0),
                      (18.0, 0.0, 17.0)]\}

    \end{Verbatim}

    \begin{center}
        \adjustimage{max size={0.6\linewidth}{0.6\paperheight}}{Clusterization_files/Clusterization_21_1.png}
    \end{center}
    { \hspace*{\fill} \\}
    \newpage
    \begin{Verbatim}[commandchars=\\\{\}]
{\color{incolor}In [{\color{incolor}10}]:} \PY{n}{try\PYZus{}clusterize}\PY{p}{(}\PY{l+m+mf}{2.4}\PY{p}{,} \PY{n}{canberra\PYZus{}distances}\PY{p}{)}
\end{Verbatim}


    \begin{Verbatim}[commandchars=\\\{\}]
\{(-0.5, 12.5, -23.0): [(0.0, 16.0, -16.0), (-1.0, 9.0, -30.0)],
 (5.83, 4.58, 7.58): [(0.0, 1.0, 1.0),
                      (0.0, 1.0, 7.0),
                      (5.0, 7.0, 4.0),
                      (0.0, 5.0, 5.0),
                      (9.0, 4.0, 5.0),
                      (7.0, 1.0, 2.0),
                      (10.0, 0.0, 19.0),
                      (0.0, 12.0, 7.0),
                      (-5.0, -4.0, 5.0),
                      (20.0, 10.0, 15.0),
                      (18.0, 0.0, 17.0),
                      (6.0, 18.0, 4.0)]\}

    \end{Verbatim}

    \begin{center}
        \adjustimage{max size={0.6\linewidth}{0.6\paperheight}}{Clusterization_files/Clusterization_23_1.png}
    \end{center}
    { \hspace*{\fill} \\}
    
    \section{Вывод}\label{ux432ux44bux432ux43eux434}

Рассмотренный метод кластеризации вполне подходит для задач
кластеризации множества образов с небольшим количеством признаков.
Оценить использованные методы вычисления расстояний при данном методе
кластеризации довольно трудно из-за малого количества данных и
отсутствия решаемой задачи как таковой.


    % Add a bibliography block to the postdoc
    
    
    
    \end{document}
