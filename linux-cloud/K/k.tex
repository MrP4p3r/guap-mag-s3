\documentclass[12pt,a4paper]{article}
% -- LuaLaTeX --

\usepackage{qsettings}

% настройки титульной страницы
\doctype{REP}

% \kafedra{Кафедра вычислительных систем и сетей}

\profrank{доц., канд. техн. наук} % проф., д.т.н.
\profname{В. В. Балберин} % А. В. Гордеев

\doctitle{\uppercase{Отчет о курсовой работе}}
\docsubject{Разработка технического задания по индивидуальной теме} % тема отчета
\docdiscip{Построение открытых облачных систем на базе ОС GNU/Linux} % название предмета (дисциплины)

\studclass{4645М} % номер группы
\studname{Г. В. Гетманенко} % имя студента

\docfooter{\the\year} % футер титульной страницы (e.g. СПб, 2015)

%\nofiles

%\directlua{ require("drawboxes")}\usepackage{atbegshi}\AtBeginShipout {\directlua{drawboxes.visual_debug()}}

% ---------- НАЧАЛО ДОКУМЕНТА ----------

\begin{document}
\makeqtitle
\setcounter{page}{2}

\section{Цель работы}

Составление техническое документации к курсовой работе.

\section{Техническое задание}

Целью работы является получение опыта по настройке веб-сервера, DNS-сервера, создания структуры домена;
деплой простого веб приложения; организация доступа к документации на readthedocs.io через
свое доменное имя; получение SSL сертификата.

Задание:

\begin{enumerate}
    \item Настроить VPS сервер под управлением ОС GNU/Linux в качестве DNS-сервера.
          Необходимо прописать в файле зоны привязки к доменным именам для доступа
        \begin{enumerate}
            \item к веб-приложению,
            \item к документации, размещенной на readthedocs.io.
        \end{enumerate}
    \item Настроить службу запуска и остановки работы веб-приложения.
          Приложение реализовано на языке Python как WSGI приложение. Поэтому для его запуска
          необходимо использовать веб-сервер WSGI приложений.
    \item Настроить доступ к веб-приложению и документации через соответствующие доменные имена.
    \item Получить SSL сертификаты для доступа к веб-приложению и документации через протокол HTTPS.
\end{enumerate}

\section{Вывод}

В процессе работы была рассмотрена методика конфигурирования
DNS сервера для своей доменной зоны, конфигурирования веб-сервера Nginx в
качестве обратного прокси на свое веб-приложение и документацию, размещенную
на readthedocs.org. Создание службы для запуска и остановки работы
внутреннего сервера WSGI приложения.


\end{document}
